\begin{abstract}
	\noindent With an increasing market share of wind energy in Europe's energy production, power regulation of wind farms becomes more important. Power regulation of a wind farm can be achieved by axial induction control and yaw control. However, behind every operating turbine a wind flow with decreased velocity and increased turbulence occurs, called a wake. Overlap of the wake on a downstream turbine causes loss of power and increased dynamical loads. 
The current control method (greedy control) only takes power into account. Previous studies have mainly focused on power maximization. This paper shows a method to regulate power output of a wind farm, while reducing the loads.  The method combines the use two types of control methods (yaw and axial induction control) to find improved wind farm control settings. 
\end{abstract}

%The market share of wind energy grows and therefore wind power regulation becomes important. Power regulation of a wind farm can be achieved by axial induction control and yaw control. However, behind every operating turbine a wind flow with decreased wind velocity and increased turbulence occurs, called a wake. Overlap of the wake on a downstream turbine causes loss of power and increased dynamical loads.  
	
	%gaining importance on the energy marked, are aspects such as optimal usage of resources as well as reliability and stability of a growing importance. Nowadays, more and more of the wind energy is generated in large wind farms. These farms have the advantages of being cheaper in installation as well as maintenance. However, the current control method(greedy control) is far from optimal, and only takes power into account. This paper combines the use of 2 other types of control methods (yaw and axial induction control) to find the optimal running procedure of a wind farm. Decreasing the damaging loads on the turbines, while better regulating the power output. Making wind energy cheaper while at the same time increasing its stability. Ensuring that wind energy is ready for the future.

