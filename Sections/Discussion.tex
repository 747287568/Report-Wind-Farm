\section{Conclusion, Discussion and Recommendations}
What have we done and how to interpret the results
\subsection{Conclusion}
Conclusion about the results, Power reduction by axial  induction and yaw control will lead to a reduction in the loads on the wind farm. 
 

\subsection{Discussion \& Recommendations}
This article, which presents the results of a multi-objective wind farm optimization through yaw control and axial induction control is bound to a number of limitations. Below these are discussed and recommendations for future work are given.
\begin{itemize}
	\item In optimization, loads that are introduced by varying the axial induction factor of a turbine are not considered. This can change the outcome of the optimization. To resolve this issue, additional loads caused by varying the axial induction factor can be investigated and stored in the lookup table.
	\item For the flow field, two recommendations can be given. FAST calculates the loads on the turbine based on a modeled flow field which does not take turbulence effects of the wake into account. Added turbulence effects can change the results of the optimization and should, as such, be included in future work. The shear constant was set for a flat ocean ground surface. To allow optimization for a wider range, different shear constants can be added to the lookup table.
	\item Although the possibility of minimizing loads by reducing power and the change of yaw and axial induction is demonstrated in this article, a global minimum in the optimization problem is not found. Several simulations with the same input settings lead to small differences in final power and load values. The loads are only minimized for the complete wind farm, not for an individual turbine. A suggestion for further research is to set a constraint for the maximal DEL value for each of the turbine in the wind farm. This also could be a solution for more identical results regarding the turbine settings. Also, the convergence of the algorithm should be investigated, to better find a global minimum.
\end{itemize}

  
  
%\item The low fidelity steady state model FLORIS is used to calculate the flow in the wind farm. Turbulence and dynamic propagation of the wake as a consequence of turbine settings, and the effect of wakes influencing each other, are not implemented. Moreover, the  turbine determines its loads only for the most overlapping wake on the turbine, for other overlapping wakes a mean value of the velocity is taken. Besides, axial induction of the turbine is not applied to the loads but only to the power production of the turbine.    

%	\item The continuous wake profile uses only the velocity in the inner wake zone and the diameter of the outer wake zone from Gebraad. \cite{Gebraad2016} The partition of the wake in several wake zones with its own diameter and velocity is not longer implemented. For a sufficient model of the wake, a continuous function with more adjustable variables compared to one Gaussian function should be considered.  

%	\item Linear interpolation out of the LUT is used to calculate the DEL values for intermediate parameter values. This can lead to a deviation in DEL values with respect to a direct calculation. For one range of calculations of the LUT some extreme values occur. These values have been smoothed. However, these values lay out of the scope of the optimization problem so it had not an effect on the optimization at all.   