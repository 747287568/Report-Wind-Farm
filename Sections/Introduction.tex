\section{Introduction}
	
\lettrine[nindent=0em,lines=3]To minimize cost of wind energy and to efficiently use resource rich locations, wind farms are constructed, characterized by a large concentration of wind turbines. An inherent side effect of wind turbine however is a wake. When wind turbines are placed in close proximity of each other in a wind farm, a wake can have negative effects. A wake forms behind an upwind wind turbine and can have severe effects on the power production and loads of the downwind turbine. 

With an increasing role of wind energy in Europe's energy production \cite{Nat2016}, it is important for wind farms to be able to meet the demands of the power system \cite{Tande2003}. Controlling the power output of a wind farm is essential because an overload of energy can decrease the stability of the power system \cite{Tande2003}. Currently, most wind farms operate based on 'greedy control', meaning that the individual turbines always try to deliver maximum power. This causes a problem when the power demand is low and the dependency on the wind energy is high, resulting in an overload of the power system. Active power control can solve this problem, by regulating the power output of a wind farm.

There are several methods of active power control in a wind farm, two of which will be discussed in this paper, that is, yaw control and axial induction control. Yaw control can be used as a method to reduce the power output of a single wind turbine and as a method to increase power output and reduce loads of a wind farm (verwijzing). A disadvantage of yawing is an increase of the load on individual turbines, and thus reducing its lifespan \cite{Zalkind2016,Kanev2017}. In addition, yawing a turbine can deflect a wake. This can be used as a method to redirect a wake from a downwind turbine, this would reduce the loads and increase power output of the downwind turbine(verwijzing). However, under unfavorable conditions it can result in an asymmetrical overlap of the wake on the downwind turbine. This can significantly increase the loads of the downwind turbine \cite{Wilson2017,Dijk2016}(nog meer verwijzingen). Therefore, it is necessary to understand the effects of yawing a turbine with respect to wake propagation. Axial induction can control the power output by varying the axial induction factor during the optimization. Loads that are introduced by axial induction will not be discussed in this paper.  As yawing of the turbine and the deflection of a wake can cause additional loads on the turbines of a wind farm, a trade of must be made between active power control and loads minimization.

Previous studies have mainly focused on optimizing the power output, rather than controlling the power output (verwijzingen naar studies die dit hebben gedaan).  Other studies focus on power optimization, while also taking the loads into account, so that an optimum is found between the two (verwijzing naar wie dit heft gedaan). The use of axial induction in optimizing power control has been marginally studied(heb hier nog geen goede bron voor kunnen vinden). Active power control through a combination of yaw control and axial induction control while minimizing loads is still novel.

This paper will focus on the optimization of active power control and loads, by means of yaw control and axial induction control. In addition, a method is developed so that on-site power control can be realized.
	
Hier nog een overzicht van wat er in paper te vinden is. Moet later geschreven worden als de rest af is.