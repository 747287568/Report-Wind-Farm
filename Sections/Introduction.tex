\section{Introduction}
\lettrine[nindent=0em,lines=3]To efficiently use wind resource rich locations and minimize the cost of wind energy, wind turbines are constructed in close proximity of each other, resulting in a wind farm. A side effect of wind turbine operation is a wake. When wind turbines are placed in close proximity of each other, wakes can have negative effects on downwind turbines. A wake is developed behind a wind turbine and can reduce the power production while increasing the loads of the downwind turbine. A wake can be described by several characteristics, such as a decreased wind velocity caused by the turbine's extraction of energy, an increased turbulent wind flow caused by the rotation of the turbine blades, and a wake deflection can occur which can lead to suboptimal power and load conditions \cite{Boersma2017, Wilson2017, Dijk2016, Fleming2014, Zalkind2016}.

With an increasing role of wind energy \textcolor{blue}{(quantify this)} in Europe's energy production new challenges arise\cite{Nat2016}. While the electrical power system becomes more dependent on wind energy, it is important for wind farms to be able to meet the demands of the power system by regulating its power output \cite{Tande2003}. Traditionally, the need for power regulation of a wind farm was not necessary, because the wind farms energy production was only a small part of the total energy production. As a result, the wind farms could always perform at maximum capacity, without the risk of overloading the power system. However, because wind energy production has increased in the recent years, and will continue to increase, regulating rather than maximizing power output becomes more important.\\ 
\indent Regulating the power output of a wind farm is essential because an under- or overload of energy can decrease the stability of the power system \cite{Tande2003}. Instability can lead to suboptimal operation of the power system, such as an energy overload \cite{Tande2003, Hansen20}. Currently, most wind farms operate based on 'greedy control', meaning that the individual turbines always try to deliver maximum power for the turbine and not taking additional loads into account introduced by power maximization. This causes a problem when, for example, the power demand is low and the energy dependency of the power system on wind energy is high. This could then result in an overload of the power system. Power regulation of a wind farm can solve this problem.

There are several methods for power regulation of a wind farm, two of which will be discussed in this paper, that is, yaw control and axial induction control. Yaw control can be used as a method to reduce the power output of a single wind turbine and as a method to increase power output and reduce loads of a wind farm \cite{Dijk2016, Wilson2017, Fleming2014, vanDijk2016}. A disadvantage of yawing is an increase of load on the yawed turbine, and thus reducing its lifespan \cite{Zalkind2016,Kanev2017}. In addition, yawing a turbine can deflect a wake. This can be used as a method to redirect a wake from a downwind turbine. This reduces the loads and increases the power output of the downwind turbine(van dijk, wilson, fleming). However, under unfavorable conditions it can result in an asymmetrical overlap of the wake on the downwind turbine. This can significantly increase the loads of the downwind turbine \cite{Wilson2017,Dijk2016}. Therefore, it is necessary to understand the effects of yawing a turbine, what this does with its wake, and how this effects the downwind turbines. 

Axial induction control influences the power output of a turbine by means of the blade pitch angle and generator torque. The axial induction factor is denoted by $a$, and reflects the relation between the decrease in wind velocity of the free stream wind and the wind velocity leaving the rotor. By varying the axial induction factor of a wind turbine, the power output can be controlled. Loads that are introduced by varying a turbine's axial induction factor will not be discussed in this paper. 
As yawing of the turbine and the deflection of a wake can cause additional loads on the turbines of a wind farm, a trade-off must be made between power regulation and loads minimization.

Previous studies have mainly focused on maximizing the power output while taking the loads into account, so that an optimum is found between the two \cite{Dijk2016, vanDijk2016, Wilson2017}. However, increased loading effects on a turbine due to yaw misalignment were not considered. Furthermore, the objective of these studies is power maximization rather than power regulation. %The use of axial induction factors for regulating power output of a wind farm has been marginally studied. 
Power regulation through a combination of yaw control and axial induction control while minimizing loads is still novel. This paper will focus on the optimization of power regulation, by means of yaw control and axial induction control, while minimizing the loads. In addition, the loading effects on a turbine caused by yaw misalignment is studied. 

Hier nog een overzicht van wat er in paper te vinden is. Moet later geschreven worden als de rest af is.
