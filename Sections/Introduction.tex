\section{Introduction}
\lettrine[nindent=0em,lines=3]
With an increasing role of wind energy production new challenges arise\cite{Nat2016}. In the EU, wind power capacity has increased sixfold, from 2,4\% market share in 2000 to 15,6\% in 2015, overtaking hydro as the third largest power generation capacity \cite{EWEA2016}. To efficiently use resource rich locations and minimize the cost of energy, wind turbines are constructed in close proximity, creating wind farms. In operation, an area of decreased wind velocity and increased turbulence occurs behind the turbine, called a wake. Overlap of the wake with a downstream turbine causes loss of power and increased dynamical loads \cite{Boersma2017, Wilson2017, Dijk2016, Fleming2014, Zalkind2016}. 

While the market becomes more dependent on wind energy, it is important for wind farms to be able to meet the demands of the power by regulating its output \cite{Tande2003}. Wind farm power regulation is needed to prevent an overloaded energy grid \cite{Hansen2014}. 

Currently, most wind farms operate based on 'greedy control', i.e. each individual turbine is set to run for maximum power production, neglecting wake interaction. Previous studies on this effect mainly focused on maximizing the power output of the wind farm while minimizing loads \cite{Dijk2016, vanDijk2016, Wilson2017}. In this paper the focus is shifted to power regulation and loads minimization. Therefore two methods will be discussed, yaw control and axial induction control. 

Yaw control can be used as a method to redirect a wake from a downstream turbine. This reduces the loads and increases the power output of the downstream turbine. A disadvantage of yawing is a possible increase of loads on the yawed turbine, and thus reducing its lifespan \cite{Zalkind2016,Kanev2017}. Furthermore, under unfavorable conditions it can result in an asymmetrical overlap of the wake on the downstream turbine, significantly increasing the loads on this turbine \cite{Wilson2017,Dijk2016}. 

Axial induction control influences the power output of a turbine by means of the blade pitch angle and generator torque. The axial induction factor reflects the relation between the free stream wind velocity and the wind velocity leaving the rotor. By varying the axial induction factor of a wind turbine, the power output can be controlled. Moreover, the axial induction factor is proportionally linked to the size of the wake and thus influences overlap between a wake and a downstream turbine.  

Hier nog een overzicht van wat er in paper te vinden is. Moet later geschreven worden als de rest af is.
