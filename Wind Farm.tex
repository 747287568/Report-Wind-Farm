%----------------------------------------------------------------------------------------
%	PACKAGES AND OTHER DOCUMENT CONFIGURATIONS
%----------------------------------------------------------------------------------------

\documentclass[twoside,twocolumn]{article}

\usepackage{blindtext} % Package to generate dummy text throughout this template 

\usepackage[sc]{mathpazo} % Use the Palatino font
\usepackage[T1]{fontenc} % Use 8-bit encoding that has 256 glyphs
\linespread{1.05} % Line spacing - Palatino needs more space between lines
\usepackage{microtype} % Slightly tweak font spacing for aesthetics

\usepackage[english]{babel} % Language hyphenation and typographical rules

\usepackage[hmarginratio=1:1,top=32mm,columnsep=20pt]{geometry} % Document margins
\usepackage[hang, small,labelfont=bf,up,textfont=it,up]{caption} % Custom captions under/above floats in tables or figures
\usepackage{booktabs} % Horizontal rules in tables

\usepackage{lettrine} % The lettrine is the first enlarged letter at the beginning of the text

\usepackage{enumitem} % Customized lists
\setlist[itemize]{noitemsep} % Make itemize lists more compact

\usepackage{abstract} % Allows abstract customization
\renewcommand{\abstractnamefont}{\normalfont\bfseries} % Set the "Abstract" text to bold
\renewcommand{\abstracttextfont}{\normalfont\small\itshape} % Set the abstract itself to small italic text

\usepackage{titlesec} % Allows customization of titles
\renewcommand\thesection{\Roman{section}} % Roman numerals for the sections
\renewcommand\thesubsection{\roman{subsection}} % roman numerals for subsections
\titleformat{\section}[block]{\large\scshape\centering}{\thesection.}{1em}{} % Change the look of the section titles
\titleformat{\subsection}[block]{\large}{\thesubsection.}{1em}{} % Change the look of the section titles

\usepackage{fancyhdr} % Headers and footers
\pagestyle{fancy} % All pages have headers and footers
\fancyhead{} % Blank out the default header
\fancyfoot{} % Blank out the default footer
\fancyhead[C]{Running title $\bullet$ May 2016 $\bullet$ Vol. XXI, No. 1} % Custom header text
\fancyfoot[RO,LE]{\thepage} % Custom footer text

\usepackage{titling} % Customizing the title section

\usepackage{hyperref} % For hyperlinks in the PDF

%----------------------------------------------------------------------------------------
%	TITLE SECTION
%----------------------------------------------------------------------------------------

\setlength{\droptitle}{-4\baselineskip} % Move the title up

\pretitle{\begin{center}\Huge\bfseries} % Article title formatting
	\posttitle{\end{center}} % Article title closing formatting
\title{Wind Farm} % Article title
\author{%
	\textsc{Niels van Duijn, Jurriaan Govers, Luuk van Hagen}\\
	\textsc{Jochem Hoorneman, Max van Leeuwen}\\%\thanks{A thank you or further information} [1ex] % Your name
	\normalsize TU Delft \\ % Your institution
	\normalsize \href{mailto:j.a.govers@gmail.com}{j.a.govers@gmail.com} % Your email address
	%\and % Uncomment if 2 authors are required, duplicate these 4 lines if more
	%\textsc{Jane Smith}\thanks{Corresponding author} \\[1ex] % Second author's name
	%\normalsize University of Utah \\ % Second author's institution
	%\normalsize \href{mailto:jane@smith.com}{jane@smith.com} % Second author's email address
}
\date{\today} % Leave empty to omit a date
\renewcommand{\maketitlehookd}{%
	\begin{abstract}
		\noindent This will be the abstract of our paper. We will show how cool our research is and how important we are. Blabla Lorem impsum.
	\end{abstract}
}

%----------------------------------------------------------------------------------------

\begin{document}
	
	% Print the title
	\maketitle
	
	%----------------------------------------------------------------------------------------
	%	ARTICLE CONTENTS
	%----------------------------------------------------------------------------------------
	
	\section{Introduction}
	
	\lettrine[nindent=0em,lines=3]{W}ith an increasing role of wind energy in Europe’s energy production (nationale energieverkenning) it is important for wind farms to be able to meet the demands of the power grid (Tande ). Controlling the power output of a wind farm is essential because an overload of energy can decrease the stability of the power system (Tande). Currently, most wind farms operate based on ‘greedy control’, meaning that the individual turbines always try to deliver maximum power. This causes a problem when the power demand is low and the dependency on the wind energy is high, resulting in an overload of the power system. Active power control can solve this problem, by regulating the power output of a wind farm.

Misschien ergens nog iets over een wake? Over wind farms algemeen?

There are several methods of active power control in a wind farm, two of which will be discussed in this paper, that is, yaw control and axial induction control. Yaw control can be used as a method to reduce the power output of a single wind turbine (verwijzing). A disadvantage of yawing is an increase of the load on the turbine, and thus reducing its lifespan (Zalkind, Kanev). In addition, yawing a turbine can result in an asymmetrical overlap of the wake on the downwind turbine. This can significantly increase the loads of the downwind turbine (Wilson, Van Dijk, Bastankah). As yawing of the turbine and the deflection of a wake can cause additional loads on the turbines of a wind farm, a trade of must be made between active power control and loads optimization. Axial induction can control the power output by varying the axial induction factor. Loads that are introduced by axial induction will not be discussed in this paper. 

Previous studies have mainly focused on optimizing the power output, rather than controlling the power output (verwijzingen naar studies die dit hebben gedaan).  Other studies focus on power optimization, while also taking the loads into account, so that an optimum is found between the two (verwijzing naar wie dit heft gedaan). The use of axial induction in optimizing power control has been marginally studied. Active power control through a combination of yaw control and axial induction control while minimizing loads is still novel.

This paper will focus on the optimization of active power control and loads by means of yaw misalignment and axial induction. In addition, a method is developed so that on-site power control can be realized.

Hier nog een overzicht van wat er in paper te vinden is. Moet later geschreven worden als de rest af is.
	\\
	\\
	What is the subject about?
	\begin{itemize}
		\item General introduction about wind energy, wind farms, wakes, and optimization strategies.
		\item Maybe problem statement here??
	\end{itemize}
	\parshape=0
	Why is this relevant?
	\begin{itemize}
		\item Problem statement
		\item What can be gained by optimization?
	\end{itemize}
	State-of-the-Art.
	\begin{itemize}
		\item What did other people do to solve this problem?
	\end{itemize}
	What is different/new about our research?
	\begin{itemize}
		\item What did we do compared to other people?
	\end{itemize}
	 What can the reader expect to find in this paper?
	\begin{itemize}
		\item Overview of paper
	\end{itemize}
	What are the most important results/conclusions of this paper?
	
	
	%------------------------------------------------
	
	\section{Methods}
	
	Describe the techniques used. 
	\begin{itemize}
		\item Here we introduce FLORIS, FAST, MLife
		\item FLORIS: why FLORIS compared to other software? How does FLORIS work? How do we use it?
		\item FAST \& Mlife:
		\item LUT: What is the LUT and why? What parameters did we choose, and why? Information about the step size of the parameters. Details in supporting info.
		\item Optimization: Game Theory
	\end{itemize}
	\blindtext % Dummy text
	
	Text requiring further explanation\footnote{Example footnote}.
	
	%------------------------------------------------
	
	\section{Results}
	
	Figures, tables with results and explanation.
	
	Discuss what can be improved
		
	\begin{table}
		\caption{Example table}
		\centering
		\begin{tabular}{llr}
			\toprule
			\multicolumn{2}{c}{Name} \\
			\cmidrule(r){1-2}
			First name & Last Name & Grade \\
			\midrule
			John & Doe & $7.5$ \\
			Richard & Miles & $2$ \\
			\bottomrule
		\end{tabular}
	\end{table}
	
	\blindtext % Dummy text
	
	\begin{equation}
		\label{eq:emc}
		e = mc^2
	\end{equation}
	
	\blindtext % Dummy text
	
	%------------------------------------------------
	
	\section{Discussion}
	
	\subsection{Subsection One}
	
	What have we done and how to interpret the results
	
	A statement requiring citation \cite{Figueredo:2009dg}.
	\blindtext % Dummy text
	
	\subsection{Subsection Two}
	
	Recommendations for further studies
	
	\blindtext % Dummy text
	
	%----------------------------------------------------------------------------------------
	%	REFERENCE LIST
	%----------------------------------------------------------------------------------------
	
	\begin{thebibliography}{99} % Bibliography - this is intentionally simple in this template
		
		\bibitem[Figueredo and Wolf, 2009]{Figueredo:2009dg}
		Figueredo, A.~J. and Wolf, P. S.~A. (2009).
		\newblock Assortative pairing and life history strategy - a cross-cultural
		study.
		\newblock {\em Human Nature}, 20:317--330.
		
	\end{thebibliography}
	
	%----------------------------------------------------------------------------------------
	
\end{document}
